\documentclass[11pt]{article}
\usepackage{fullpage}

\title{Natural User Interface and Virtual Reality Integration\\in Video Games\\{\small Last Revised: \today}}
\author{\small\begin{tabular}{c c c c} 
		Alexandre Wimmers 		& Mario Yepez 			& Timothy Tong			& Valerie Gadjali\\
		adwimmers@ucdavis.edu 	& myepez@ucdavis.edu 	& tktong@ucdavis.edu	& vgadjali@ucdavis.edu
	\end{tabular}}
\date{}


\begin{document}

\maketitle

\section{Background}

\subsection{Natural User Interface}

\paragraph{} Natural user interface is the ability to use an application without the need for any controller. In this case, the players own body or parts of the body becomes the controller. The concept of natural user interface first appeared in 1990, but the first popular usage for video games appeared in 2010 with the release of the Microsoft Kinect. Traditional keyboards and mice can only interact with a monitor or any visual interface in a two-dimensional sense. However, with a natural user interface it becomes possible to interact in three dimensions.

\subsection{Virtual Reality}

\paragraph{} Virtual reality is an immersive experience within a computer-simulated environment. First conceptualized in 1930, but only recently has it become a popular with the Oculus Rift Development Kit 1. It provides the user a complete immersion and a new perspective into the virtual world. Whereas a monitor restricts a user to only view in two dimensions, virtual reality allows the user to see the world in three dimensions.

\subsection{Unity Game Engine}

\paragraph{} Unity is the game engine of choice for this project. Developed by Unity Technologies, Unity includes both a game engine and an integrated development environment. There are a variety of factors that led to the adoption of Unity, such as portability. Games built on Unity are able to run on multiple platforms; this allows for game development regardless of operating system. A host of plug-ins and libraries are also available for Unity such as the A* Pathfinding Project and Photon, which will prove to be useful in development of this project. The most notable plug-in and library is the Oculus Rift and Leapmotion compatibility. Other game engines require workarounds while Unity makes integration simple. 

\section{Goal}

\paragraph{} Take an existing video game genre and develop a video game integrating both natural user interface and virtual reality.

\section{Game Genres}

\subsection{Real-Time Strategy (RTS)}

\paragraph{} As a subcategory of strategy video games, real-time strategy is a type of war game that requires you to manage resources and fight with an opponent in real time. Unlike turn-based games, RTS games require quick reaction and thinking to succeed. The term real-time strategy first appeared in 1982 in a game called Cytron Masters by Dani Bunten Berry. 

\subsubsection{Pros}

\begin{itemize}
	\item True three dimensional real-time strategy.
	\item "Cool" factor for in-person perspective with combat.
\end{itemize}

\subsubsection{Cons}

\begin{itemize}
	\item Reduced precision in controls due to technology.
	\item In-person perspective has no objective gain in competitive gameplay, but rather impedes player. Zooming out to bird's eye view while aligning to one of three axis will provide a much more effective perspective which defeats the purpose of certain technologies in this game.
	\item 
\end{itemize}

\subsection{Castle Defense}

\section{Implementation}

\subsection{Genre of Choice}

\subsection{}

\end{document}