\documentclass[11pt]{article}
\usepackage{fullpage}

\title{Natural User Interface and Virtual Reality Integration\\in Video Games\\{\small Last Revised: \today}}
\author{\small\begin{tabular}{c c c c} 
		Alexandre Wimmers 		& Mario Yepez 			& Timothy Tong			& Valerie Gadjali\\
		adwimmers@ucdavis.edu 	& myepez@ucdavis.edu 	& tktong@ucdavis.edu	& vgadjali@ucdavis.edu
	\end{tabular}}
\date{}

\usepackage{url}
\usepackage{array}
\usepackage{float}

\newcolumntype{L}[1]{>{\raggedright\let\newline\\\arraybackslash\hspace{0pt}}m{#1}}
\newcolumntype{C}[1]{>{\centering\let\newline\\\arraybackslash\hspace{0pt}}m{#1}}

\begin{document}
	
\maketitle

\section{Summary}
Castle Defense aims to provide the user with a fun and enjoyable gaming experience for everyone. The game makes use of Virtual Reality and a Natural User Interface to provide this experience by using emerging technologies such as the Oculus Rift Dk2 and Leapmotion. Should any problems or questions arise, feel free to contact us at \url{castledefensegame@gmail.com} 


\section{Resources Required}

\subsection{Hardware Requirements}

\begin{table}[h]
	\renewcommand*{\arraystretch}{1.5}
	\centering
	\begin{tabular}{| l | l |}
		\hline
		Leap Motion			&	\url{https://leapmotion.com/}\\
		\hline
		Oculus Rift DK2		&	\url{https://oculus.com/dk2}\\
		\hline
	\end{tabular}
\end{table}

\subsection{Software Requirements}

\begin{table}[h]
	\renewcommand*{\arraystretch}{1.5}
	\centering
	\begin{tabular}{| c | c |}
		\hline
		Leap Motion SDK		&	2.2.2.24469\\
		\hline
		Oculus Rift SDK		&	0.4.4 Beta\\
		\hline
		Unity				&	4.6.1\\
		\hline
		Unity Test Tools	&	1.5\\
		\hline
	\end{tabular}
\end{table}


\section{Build Instructions}

\begin{enumerate}
	\item Checkout the Bitbucket repo: \url{git@bitbucket.org:ecs193/project.git}
	\item Open the project in Unity
	\item Optimize the build for the Oculus Rift by going to Prefences$\rightarrow$Oculus Vr and checking `Optimize Builds for Rift'
	\item Build the game using the default settings provided by Unity. 
	\item Run \textit{CastleDefense\_DirectToRift.exe}.
\end{enumerate}

\section{Issue Reporting Procedure}

\begin{enumerate}
	\item File an email report to \url{castledefensegame@gmail.com} with subject line ``[ECS 193 --- Issue]".
	\item Provide a brief description of the issue.
	\item If possible, provide screenshots and steps to reproduce the error.
\end{enumerate}

\section{Functional Testing}

\subsection{Unit Testing}

\begin{table}[H]
	\renewcommand*{\arraystretch}{1.5}
	\centering
	\begin{tabular}{| c | c | C{6.2cm} | L{4cm} |}
		\hline
		Class/Entity	
			&	Method	
			&	Assertion	
			&	Description\\
		\hline
		Castle			
			&	Init()	
			&	CurrentHealth == MaxHealth	
			&	Assert that the castle starts with max health.\\
		\hline
			&	LoseHealth()	
			&	CurrentHealth == (MaxHealth - \( \Delta \))	
			&	Assert that the castle's health depletes and depletes by a correct amount.\\
		\hline
			&	GameOver()	
			&	CurrentHealth == 0 \newline
				GameOver.GUI == Activated	
			&	Assert that the castle health is depleted and that the end game GUI has activated.\\
		\hline
		Enemy
			&	Init()
			&	CurrentHealth == MaxHealth \newline
				Target == Castle
			&	Assert that the enemy spawns with max health and that the enemy target is the player castle.\\
		\hline
			&	LoseHealth()
			&	CurrentHealth == (MaxHealth - \( \Delta \))	
			&	Assert that the castle's health depletes and depletes by a correct amount.\\
		\hline
			&	Death()
			&	CurrentHealth == 0 \newline
				Speed == 0 \newline
				Verify(DeathAnimation).Called(1) \newline
				Verify(this).deleted \newline
			& Assert that the enemy health is 0. Assert that the enemy is no longer moving and that the death animation was called. Then assert that the enemy entity was deleted.\\
		\hline
		Enemy Manager
			&	Spawn()		
			&	SpawnPoints[] != NULL \newline
				SpawnType != NULL \newline
				Verify(Enemy.Spawn()).Called(1)
			&	Assert that spawn points and spawn type is not NULL. Assert that an enemy was spawned or instantiated.\\
		\hline
			&	Spawn()
			&	canSpawn == FALSE \newline
				Verify(Enemy.Spawn()).Called(0)
			& 	Assert that if the enemy manager is disabled that an enemy is not instantiated.\\
		\hline
	\end{tabular}
\end{table}

\subsection{Integration Testing}

\begin{table}[H]
	\renewcommand*{\arraystretch}{1.5}
	\centering
	\begin{tabular}{| c | c | C{6.2cm} | L{4cm} |}
		\hline
		Class/Entity	
			&	Method	
			&	Assertion	
			&	Description\\
		\hline
		Castle			
			&	GameOver()	
			&	CurrentHealth == 0 \newline
				GameOver.GUI == Activated \newline
				EnemyManager.canSpawn == FALSE \newline
				Enemy.disabled == TRUE
			&	Assert the same requirements as unit testing. Also assert that the enemy manager spawn and enemies are disabled.\\
		\hline
		Enemy
			&	Death()
			&	CurrentHealth == 0 \newline
				Speed == 0 \newline
				Verify(DeathAnimation).Called(1) \newline
				Verify(this).deleted \newline
				Player.Score++
			& Assert the same requirements as unit testing. Also assert that the player's score increases.\\
		\hline
	\end{tabular}
\end{table}

\section{Non-Functional Testing}

\subsection{Security}

Not applicable. This product does not contain client sensitive data. The issue of potential hacks into the game is a considerable issue. All Unity games execute through the Unity platform similarly to how Java executes in the Java Virtual Machine. By default, there is ``natural" security through this platform. Therefore, if our game was to be compromised, then the issue persists to the entire Unity platform, which falls under the jurisdiction of the Unity Security Team.

\subsection{Fault Tolerance}

Not applicable. The game does not perform any form of persistence.

\subsection{Performance Testing}

Build and run the game. Use the frame-per-second (FPS) counter to poll an average FPS. An average FPS below 30 is unacceptable. 

\end{document}