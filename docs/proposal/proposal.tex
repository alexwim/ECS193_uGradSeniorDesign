\documentclass[11pt]{article}
\usepackage{fullpage}

\title{Natural User Interface and Virtual Reality Integration\\in Video Games\\{\small Last Revised: \today}}
\author{\small\begin{tabular}{c c c c} 
		Alexandre Wimmers 		& Mario Yepez 			& Timothy Tong			& Valerie Gadjali\\
		adwimmers@ucdavis.edu 	& myepez@ucdavis.edu 	& tktong@ucdavis.edu	& vgadjali@ucdavis.edu
	\end{tabular}}
\date{}

\begin{document}

\maketitle

\renewcommand{\arraystretch}{1.4}

\section{Introduction}

\paragraph{} One of the current issues in the gaming industry, specifically in real-time strategy (RTS) gaming, is figuring out a way to provide the user with more a immersive and enjoyable experience. The standard RTS game implements multiple dimensions by a layered style of interactions. Two layers: ground and sky. Units exists in either layers and most units can only interact with at most one layer. This emulates a three dimensional strategy game, but is not truly three dimensional. This is the initial question -- is it possible to have a truly three dimensional RTS game?

\paragraph{} Based on the emergence of virtual reality and natural user interface technologies, the challenge may very well be answerable. The traditional keyboard and mouse with a monitor is incapable of interacting with three dimensional space easily. The only possible way is to keep one axis constant and then move along the other two. A natural user interface, however, receives input in three-dimensional space from the physical world. Virtual reality is then used to help aid the player immersion and camera perspective. The question now becomes  -- is it possible to integrate virtual reality and natural user interface to create a truly three dimensional RTS game?

\section{Problem Statement}

Design and develop a true three dimensional RTS game integrating virtual reality and a natural user interface.

\section{Technologies}

\begin{itemize}
	\item Virtual Reality: Oculus Rift
	\item Natural User Interface: Leap Motion
	\item Game Engine: Unity
\end{itemize}

\pagebreak

\section{Core (Primary) Deliverables}

\subsection{Interfaces}

\begin{itemize}
	\item Display the scene in virtual reality interface.
	\item Scene camera rotates as the virtual reality interface rotates.
	\item Receive input from a natural user interface and perform the corresponding actions associated with each gesture.
	\begin{itemize}
		\item Camera Zoom
		\item In-Game Pause
		\item Menu Interactions
		\item Unit Selection (Single Selection, Group Selection)
		\item Unit Movement (Point-and-Click)
		\item Unit Actions (Attack, Construct)
	\end{itemize}
\end{itemize}

\subsection{Game}

\begin{itemize}
	\item Multiplayer Connectivity with Another Player
	\item Two Spaceships
	\begin{itemize}
		\item Mothership
		\item Fighter
	\end{itemize}
\end{itemize}

\subsection{Misc.}

\begin{itemize}
	\item Main Menu
	\item Network Lobby
	\item Minimal HUD
\end{itemize}

\section{Secondary Deliverables}

\subsection{Interface}

\begin{itemize}
	\item Unit Actions (Repair, Allocate Internal Resources, Replace/Upgrade Modules)
\end{itemize}

\subsection{Game}

\begin{itemize}
	\item Resources
	\item Variety of Spaceships
	\begin{itemize}
		\item Cruiser
		\item Frigate
	\end{itemize}
	\item Spaceship Module Framework
	\begin{itemize}
		\item Engine
		\item Matter Generator
		\item Power Generator
		\item Radar
		\item Shields
		\item Stealth
	\end{itemize}
	\item Spaceship Weapons Customization
	\begin{itemize}
		\item Laser
		\item Missile
	\end{itemize}
\end{itemize}

\subsection{Misc.}

\begin{itemize}
	\item 3D Models
	\item Expanded HUD
	\item GUI
	\item Sound Effects
	\item Music
\end{itemize}

\section{Timeline}

\begin{table}[h]
\centering
\begin{tabular}{l  c  l}
	\hline
	Winter 2015\\
	\hline
	Week 4	&	Jan. 26	&	Specification (Version 1)\\
			&			&	Design (Version 1)\\\\
	Week 8	&	Feb. 23	&	Testing Documentation\\\\
	Week 9	&	Mar. 2	&	Prototype\\\\
	Week 10	&	Mar. 9	&	Specification (Version 2)\\
			&			&	Design (Version 2)\\\\\\
	\hline
	Spring 2015\\
	\hline
	Week 5	&	Apr. 27	&	Beta System\\\\
	Week 6	&	May 4	&	Final Testing Documentation\\
			&			&	Preliminary Delivery\\\\
	Week 9	&	May 25	&	Final System\\\\
	Week 11	&	Jun. 8	&	Presentation
\end{tabular}
\end{table}

\end{document}